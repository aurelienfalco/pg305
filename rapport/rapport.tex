\documentclass[10pt]{article}
\usepackage[utf8]{inputenc}
\usepackage[T1]{fontenc}
\usepackage{graphicx}
\usepackage{appendix}
\usepackage[french]{babel}
\usepackage{fancyhdr}
\usepackage{geometry}
\geometry{hmargin=3cm,vmargin=3.5cm}
\usepackage{enumitem}
\usepackage{listings}
\usepackage{subcaption}
\usepackage{amsmath}
\usepackage{animate}
\graphicspath{{./img/}}
\lstset{ %
  backgroundcolor=\color{white},   % choose the background color; you must add \usepackage{color} or \usepackage{xcolor}
  basicstyle=\small,               % the size of the fonts that are used for the code
  breakatwhitespace=false,         % sets if automatic breaks should only happen at whitespace
  breaklines=true,                 % sets automatic line breaking
  captionpos=b,                    % sets the caption-position to bottom
  commentstyle=\color{blue},      % comment style
%  escapeinside={\%*}{*)},          % if you want to add LaTeX within your code
  extendedchars=true,              % lets you use non-ASCII characters; for 8-bits encodings only, does not work with UTF-8
  frame=single,                    % adds a frame around the code
  keepspaces=true,                 % keeps spaces in text, useful for keeping indentation of code (possibly needs columns=flexible)
%  keywordstyle=\color{blue},       % keyword style
  language=C,                    % the language of the code
  numbers=left,                    % where to put the line-numbers; possible values are (none, left, right)
  numbersep=5pt,                   % how far the line-numbers are from the code
  numberstyle=\tiny\color{black},  % the style that is used for the line-numbers
  rulecolor=\color{black},         % if not set, the frame-color may be changed on line-breaks within not-black text (e.g. comments (green here))
  showspaces=false,                % show spaces everywhere adding particular underscores; it overrides 'showstringspaces'
  showstringspaces=false,          % underline spaces within strings only
  showtabs=false,                  % show tabs within strings adding particular underscores
  stepnumber=1,                    % the step between two line-numbers. If it's 1, each line will be numbered
  tabsize=2,                       % sets default tabsize to 2 spaces
}
\usepackage{hyperref}
\hypersetup{colorlinks=true}

\pagestyle{fancy}

\renewcommand{\sectionmark}[1]{ \markright{#1}{} }

\lhead{}
\rhead{}
\lfoot{ENSEIRB-MATMECA}
\rfoot{PRCD}
\renewcommand{\headrulewidth}{0.pt}
\renewcommand{\footrulewidth}{0.4pt}


\begin{document}
%\tableofcontents
%\newpage

\thispagestyle{empty}


%\begin{center}
%\includegraphics{enseirb_inp.png}
%\end{center}
%
%\vspace{\stretch{1}}
\hrule
\begin{flushleft}
\Huge{\textbf{Projet PG305}}\\
\textit{Craquage de mot de passe}
\end{flushleft}
\begin{flushright}
\huge\textbf{Rapport}\\
\end{flushright}
\hrule

\vspace{80pt}
\noindent\textbf{Élève :}
\emph{Aurélien Falco, Alexandre Honorat}\\
\\
\noindent\textbf{Responsable :}
\emph{Olivier Coulaud}\\


\vspace{60pt}
\normalsize
\begin{center}
  Troisième année, filière informatique, option PRCD\\
  Date : \today\\
  Enseirb-Matmeca
\end{center}
\vspace{50pt}

%% \section*{Introduction} % (fold)
\addcontentsline{toc}{section}{Introduction}
\label{sec:introduction}
Le but du projet était d'implémenter une recherche de mot de passe par force brute, répartie sur plusieurs processus. Dans ces processus la charge serait sur plusieurs threads OpenMP.



\section{Introduction} % (fold)
\label{sec:introduction}
Le but du projet était d'implémenter une recherche de mot de passe par force brute, répartie sur plusieurs processus. Dans ces processus la charge serait sur plusieurs threads OpenMP.


\section{Recherche de mot de passe} % (fold)
\label{sec:recherche_de_mot_de_passe}

Le programme devait effectuer une reracherche de mot de passe par force brute. L'utilisateur lance tout d'abord un programme appelé master (maître). Celui-ci lance $p-1$ processus appelés slave (esclave). Chaque processus esclave lance à son tout $t+1$ threads. Parmi ces threads, un servira aux communications avec le maître et les autres effectueront des tâches. Ces tâches sont en fait des intervalles de mots à vérifier. 

TODO: terminaison

\section{Organisation} % (fold)
\label{sec:organisation}

Le code a été segmenté pour permettre une lecture plus aisée de l'ordre dans lequel s'effectuent les diverses opérations. Le code du maître se trouve donc dans un fichier séparé de celui des esclaves. Le fonctionnement du thread de communication est regroupé dans une fonction séparée de celle des threads de travail. Des fonctions permettent de calculer le mot suivant dans l'ordre que nous avons fixé et de vérifier si un mot est le mot de passe. L'ordre en question est ici lexicographique par taille, c'est-à-dire que nous comparons d'abord tous les mots d'une certaine longueur, dans l'ordre lexicographique, puis nous passons aux mots d'une lettre de plus.

Un intervalle est composé de deux éléments: un mot de départ et le nombre de mots à vérifier. Ainsi, chaque thread, peut vérifier tout un intervalle, en utilisant la fonction calculant le mot suivant.

\section{Exécution \& tests} % (fold)
\label{sec:execution}

A l'exécution l'utilisateur peut passer certains paramètres au programme. Ceux-ci sont les suivants :
\begin{itemize}
	\item $p$ : nombre de processus total. $p-1$ esclaves seront créés ;
	\item $t$ : nombre de threads de travail. Au total, $t+1$ threads sont créés, en comptant le thread de communication ;
	\item $r$ : longueur maximum des mots à vérifier ;
	\item $m$ : le mot de passe.
\end{itemize}


\section{Performances} % (fold)
\label{sec:perf}

Une série de tests est lancée sur différentes tailles de mots

\begin{figure}[H]
\centering
% \includegraphics[width=0.8\textwidth]{stats.png}
\caption{Résultats obtenus}
\label{fig:stats}
\end{figure}


% section \ (end)


%% \input{X-Conclusion.tex}


\end{document}
